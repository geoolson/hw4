\documentclass[12pt]{article}
\usepackage{listings}
\usepackage{amsmath}
\usepackage{graphicx}
\usepackage{hyperref}
\usepackage[latin1]{inputenc}

\title{CS350 Homework 3}
\author{Geoffrey Olson Jr.}
\date{02/27/2019}

\begin{document}
\maketitle
\textbf{Problem 1: Prim's Algorithm}
For implementing Prim's Algorithm I referenced section 9.1 in the Third Edition of "The Design and Analysis of Algorithms" and a step by step visualization and explanation of Prim's algorithm  at:
\newline
\url{https://www.youtube.com/watch?v=cplfcGZmX7I}

For this algorithm I chose python for the ease of building a graph. Python's built in hash table ADT called the dictionary and the list ADT makes building a graph straightforward. The dictionary has an average case of O(1) access time and the list ADT has an access time of O(1) in python.
\url{https://wiki.python.org/moin/TimeComplexity}
These access times are beneficial for minimizing the complexity of this implementation. Using dictionary and list I was able to build an adjacency list to represent the graph.

For the output the python script prints the minimum spanning tree into a dot file format. This file can be compiled into a visualized graph with the weighted edges labeled on the linux servers via : 
\begin{lstlisting}
$ python3 hw4.py > mst.dot
$ dot -Tpdf mst.dot -o mst.pdf
\end{lstlisting}

city-pairs.txt must also be in the same directory to run the script or you can use a different file if you provide the name of the file as the first argument and the root city as the second argument into the script like so:
\begin{lstlisting}
$ python3 hw4.py [filename] [root city] > mst.dot
\end{lstlisting}

The most challenging part of this assignment was learning how the algorithm works. Once I understood the algorithm most of the coding was straight-forward.

\textbf{Examples}
Here are some examples outputs of the script. These are the raw dot files:

\textbf{root :} Portland
\begin{lstlisting}
graph prim {
"Portland" -- "Gresham" [label=14];
"Portland" -- "Forest.Grove" [label=23];
"Portland" -- "Newberg" [label=23];
"Newberg" -- "McMinnville" [label=14];
"Newberg" -- "Woodburn" [label=19];
"Woodburn" -- "Salem" [label=17];
"Salem" -- "Albany" [label=24];
"Albany" -- "Corvallis" [label=11];
"Corvallis" -- "Eugene" [label=40];
"Corvallis" -- "Newport" [label=53];
"Eugene" -- "Springfield" [label=4];
"Forest.Grove" -- "Tillamook" [label=52];
"Newport" -- "Florence" [label=50];
"Florence" -- "Coos.Bay" [label=48];
"Tillamook" -- "Astoria" [label=66];
"Springfield" -- "Roseburg" [label=68];
"Roseburg" -- "Grants.Pass" [label=68];
"Grants.Pass" -- "Medford" [label=29];
"Medford" -- "Ashland" [label=12];
"Ashland" -- "Klamath.Falls" [label=64];
"Gresham" -- "The.Dalles" [label=73];
"The.Dalles" -- "Redmond" [label=114];
"The.Dalles" -- "Pendleton" [label=125];
"Redmond" -- "Bend" [label=16];
"Pendleton" -- "La.Grande" [label=52];
"La.Grande" -- "Baker.City" [label=44];
"Baker.City" -- "Ontario" [label=72];
"Ontario" -- "Burns" [label=130];
}
\end{lstlisting}
\textbf{root :} Bend
\begin{lstlisting}
graph prim {
"Bend" -- "Redmond" [label=16];
"Bend" -- "Burns" [label=130];
"Redmond" -- "The.Dalles" [label=114];
"The.Dalles" -- "Gresham" [label=73];
"The.Dalles" -- "Pendleton" [label=125];
"Gresham" -- "Portland" [label=14];
"Portland" -- "Forest.Grove" [label=23];
"Portland" -- "Newberg" [label=23];
"Newberg" -- "McMinnville" [label=14];
"Newberg" -- "Woodburn" [label=19];
"Woodburn" -- "Salem" [label=17];
"Salem" -- "Albany" [label=24];
"Albany" -- "Corvallis" [label=11];
"Corvallis" -- "Eugene" [label=40];
"Corvallis" -- "Newport" [label=53];
"Eugene" -- "Springfield" [label=4];
"Forest.Grove" -- "Tillamook" [label=52];
"Newport" -- "Florence" [label=50];
"Florence" -- "Coos.Bay" [label=48];
"Tillamook" -- "Astoria" [label=66];
"Springfield" -- "Roseburg" [label=68];
"Roseburg" -- "Grants.Pass" [label=68];
"Grants.Pass" -- "Medford" [label=29];
"Medford" -- "Ashland" [label=12];
"Ashland" -- "Klamath.Falls" [label=64];
"Pendleton" -- "La.Grande" [label=52];
"La.Grande" -- "Baker.City" [label=44];
"Baker.City" -- "Ontario" [label=72];
}
\end{lstlisting}
\textbf{root :} Albany
\begin{lstlisting}
graph prim {
"Albany" -- "Corvallis" [label=11];
"Albany" -- "Salem" [label=24];
"Salem" -- "Woodburn" [label=17];
"Woodburn" -- "Newberg" [label=19];
"Newberg" -- "McMinnville" [label=14];
"Newberg" -- "Portland" [label=23];
"Portland" -- "Gresham" [label=14];
"Portland" -- "Forest.Grove" [label=23];
"Corvallis" -- "Eugene" [label=40];
"Corvallis" -- "Newport" [label=53];
"Eugene" -- "Springfield" [label=4];
"Forest.Grove" -- "Tillamook" [label=52];
"Newport" -- "Florence" [label=50];
"Florence" -- "Coos.Bay" [label=48];
"Tillamook" -- "Astoria" [label=66];
"Springfield" -- "Roseburg" [label=68];
"Roseburg" -- "Grants.Pass" [label=68];
"Grants.Pass" -- "Medford" [label=29];
"Medford" -- "Ashland" [label=12];
"Ashland" -- "Klamath.Falls" [label=64];
"Gresham" -- "The.Dalles" [label=73];
"The.Dalles" -- "Redmond" [label=114];
"The.Dalles" -- "Pendleton" [label=125];
"Redmond" -- "Bend" [label=16];
"Pendleton" -- "La.Grande" [label=52];
"La.Grande" -- "Baker.City" [label=44];
"Baker.City" -- "Ontario" [label=72];
"Ontario" -- "Burns" [label=130];
}
\end{lstlisting}
\end{document}

